 
% \newcommand{\companynamevalue}{Hudson River Trading}
% \newcommand{\positionnamevalue}{Software Engineer (C++) – 2024 Grads}

% \newcommand{\jobboardnamevalue}{University of Toronto Career Learning Network}
\newcommand{\jobboardnamevalue}{Mars Job Board
}
% \newcommand{\jobboardnamevalue}{LinkedIn
% }
% \newcommand{\jobboardnamevalue}{WaterlooWorks Job Board}


\documentclass[]{cover}
\usepackage{fancyhdr}
\usepackage{fontawesome}
\pagestyle{fancy}

\usepackage[colorlinks = true,
    linkcolor = blue,
    urlcolor  = blue,
    citecolor = blue,
    anchorcolor = blue]{hyperref}

\fancyhf{}
 
\rfoot{Page \thepage \hspace{1pt}}
\thispagestyle{empty}
\renewcommand{\headrulewidth}{0pt}
\begin{document}

%%%%%%%%%%%%%%%%%%%%%%%%%%%%%%%%%%%%%%
%
%     TITLE NAME
%
%%%%%%%%%%%%%%%%%%%%%%%%%%%%%%%%%%%%%%
\namesection{Leon}{Yao}{ \urlstyle{same}
\href{github.com/leoncyao}{\faGithub \hspace{1pt} github.com/leoncyao},
\href{https://leoncyao.github.io/blog/}{leoncyao.github.io/blog/}, \\ 
\href{mailto:leoncyao@gmail.com}{leoncyao@gmail.com}, 
\href{https://www.linkedin.com/in/leon-yao}{\faLinkedin \hspace{1pt} linkedin.com/in/leon-yao},
}

%%%%%%%%%%%%%%%%%%%%%%%%%%%%%%%%%%%%%%
%
%     MAIN COVER LETTER CONTENT
%
%%%%%%%%%%%%%%%%%%%%%%%%%%%%%%%%%%%%%%
\hfill

\begin{minipage}[t]{0.5\textwidth} 
\companyname{\companynamevalue}
\companyaddress{\positionnamevalue}
\end{minipage}
\begin{minipage}[t]{0.49\textwidth} 
\currentdate{\today}
\end{minipage}

\lettercontent{Dear Hiring Manager,}

% The first paragraph of your job application letter should include information on why you are writing. Mention the job you are applying for and where you found the position. If you have a contact at the company, mention the person's name and your connection here.

\lettercontent{I was delighted to come across the \positionnamevalue \hspace{1pt} posting at \companynamevalue. I believe that my knowledge and skills make me an ideal candidate for your esteemed company.
}

% \lettercontent{
%     I am currently in the Computational Mathematics Master's Program at the University of Waterloo, having completed my Undergraduate degree at the University of Toronto in Computer Science and Mathematics. From the Computer Science courses I have taken, I have become skilled in programming, through various assignments and projects. 
%     % From the mathematics courses I have taken, I have gained knowledge in logic and set theory. Furthermore, I have become skilled at problem solving through the variety of mathematics courses I have taken. 
%     % One example of this was when I was trying to debug a program to calculate a mathematical invariant for knots. I found that listing my algorithm out in steps and going over each step as if it was a line in a mathematical proof helped me solve the problem.  
% }

   
% The next section of your cover letter should describe what you have to offer the company. Make strong connections between your abilities and the requirements listed in the job posting. Mention specifically how your skills and experience match the job. Expand on the information in your resume.
% Try to support each statement you make with a piece of evidence. Use several shorter paragraphs or bullets rather than one large block of text, which can be difficult to read and absorb quickly.
%For example 

% Using the knowledge and skills I have acquired in my <DEGREE PROGRAM> combined with my technical abilities I believe I can provide significant value to <COMPANY> while providing me invaluable experience to <PREPARE FOR OR FURTHER CAREER>.
% <RELEVANT LIFE STORY/ WORK/ PROJECT >
% In own projects, my coursework, work as a <PREVIOUS JOB TITLE> I have applied knowledge of <SUBJECT MATTER>.
% <LIST OF SMART SOUNDING SKILLS AND TOOLS>.
% Combining these tools with my interests in <FIELD> issues, I hope to help the <DEPARTMENT> answer questions about the <SUBJECT> behind <FIELD> projects.


% \lettercontent{
%     I am currently in my fourth year studying Mathematics and Computer Science at the University of Toronto. From the Computer Science courses I have taken, I have become skilled in programming, especially in Python, through various assignments and projects. From the mathematics courses I have taken, I have gained knowledge in logic and set theory. Furthermore, I have become skilled at problem solving through the variety of mathematics courses I have taken. I was recently trying to debug a program to calculate a mathematical invariant for knots. I found that listing my algorithm out in steps and going over each step as if it was a line in a mathematical proof helped me solve the problem.   
% }
% \lettercontent{
% I graduated from the University of Toronto in 2021 for Computer Science. I am currently a Masters student at the University of Waterloo studying Mathematics, and I will graduate in April 2023. I think my skills would make me a good fit at Xesto
% }

\lettercontent{
In the Summer of 2022, I started an 8 month internship at \href{https://www.ecopiatech.com/}{Ecopia}, as part of studies at the University of Waterloo. 
% Throughout my internship I worked on various projects related to Computer Vision. 
One project I worked on was to blend satellite images together, as Ecopia uses satellite data to create 3D models of cities. The program I developed in C++ utilized various image blending algorithms to seamlessly merge images at efficient speeds. This experience, along with others during my internship, significantly enhanced my debugging and problem-solving skills. Working in a small, growing company like Ecopia demanded independent learning and problem resolution, as my colleagues were often engaged in their own projects.
}

% \lettercontent{
% From May 2022 to December 2022, I worked at a company called \href{https://www.ecopiatech.com/}{Ecopia} as a Computer Vision Intern. One of the projects I worked on was creating 3D models of cities. These models had many applications; one such application was in the scenario of a person calling 911 from a tall building. Using data from the model of the building, the information of which room the person called from could be sent to emergency service workers, to save time when rescuing the person. 
% My tasks included training machine learning models to reconstruct 3D buildings from satellite images, and implementing algorithms to blend 2D map images together. 
% }

\lettercontent{
From March 2023 to November 2023, I served as a Full Stack Software Developer at \href{https://www.getencircle.com/}{Encircle}. In this role, I contributed to various aspects of the company's products, spanning web and mobile app development, as well as backend tasks. I honed skills in Agile Development, optimized SQL requests in Python, and engaged in mobile app development using Java/Kotlin. Dealing with production issues taught me valuable lessons in problem mitigation for clients, ensuring minimal service interruption.
}

% \lettercontent{
% During my undergraduate studies at the University of Toronto, I had the opportunity to collaborate with other students to design a game called \href{https://neonleon123.itch.io/Viewshift}{ViewShift} for a game design course. The game was a 3D puzzle platformer made in Unity with C\#, and was focused around a core mechanic of rotating the camera to find new areas of traversal. 
% We worked on the project throughout the term, with students from different colleges and different disciplines around Toronto. It was very rewarding to first brainstorm ideas for assets and levels together, and then see my teammates create models and music, and to finally implement those assets and integrate them with the game's mechanics. 
% % I hope to work with similarly diverse multidisciplinary teams at Interaptix Augmented Reality. 
%     % I also found it satisfying to apply my strong mathematical background to implement mechanics such as rotations and camera perspectives, which have a basis in linear algebra. 
% }


% \lettercontent{
%     Another game I have made was a 3D Chess Game in Unity. This was one of the first games I made, and at the time I had only started to learn about Software Design topics such as version control and networking. It was challenging to implement local area network play with sockets, but it was made all the more worthwhile when I was able to show my friends and play together. In later years of undergraduate studies, I took several computer science courses such as a Software Design and Software Tools, where I was able to become proficient with these topics. One of my favorite courses was physics based simulation course where we used C++ and OpenGL to simulate phenomenon such as rigidbody and 3D mass springs systems. I expect to obtain a deeper understanding of similar topics in my Computational Mathematics Master's Degree at the University of Waterloo.
% }
% General Software Dev
% \lettercontent{
%     I completed my undergraduate degree in Mathematics and Computer Science at the University of Toronto. From the Computer Science courses I have taken, I have become skilled in programming, and have gained valuable software development skills such as using version control systems like git, user interface development, working in a team environment and software testing, as well as practical experience in fields such as machine learning and statistics through assignments. I have learned various languages including C++, Java and Javascript languages through schoolwork and my own personal projects. Furthermore, I have become skilled at problem solving through the variety of mathematics courses I have taken. I was recently trying to debug a program to calculate a mathematical invariant for knots. I found that listing my algorithm out in steps and going over each step as if it was a line in a mathematical proof helped me solve the problem.   
% }

% OS Software Dev
\lettercontent{
I hold an undergraduate degree in Mathematics and Computer Science from the University of Toronto. My coursework, particularly in Operating Systems and System Tools, has equipped me with strong programming skills. Projects, such as designing and implementing a file system, have deepened my understanding of UNIX and system/processor performance. This experience provided insight into the complexity of everyday software, including cloud and networking services.
}
% \lettercontent{
%     In Summer 2019, I worked as a Lab programmer for Budding Minds Lab at the University Of Toronto. During my time there, I developed iPad games that tested child brain function. I programmed in Swift and used XCode to design user interfaces. It was a rewarding experience to coordinate with psychology students to implement psychological tests.
% }

% \lettercontent{
%     In Summer 2020, through the NSERC program, I worked as a research assistant at the University Of British Columbia for Professor Liam Watson in the department of mathematics, where I worked on creating an online table of knots with properties listed. Knots arise in areas such as quantum physics and biology, where DNA strands may become knotted, and having a table with the invariants listed helps mathematicians distinguish knots easily. The invariants were calculated using Python and the website was created using HTML and Javascript. 
% }


% \lettercontent{
%     I have also worked as a Web Application Programmer at \href{https://www.radlab.zone/}{radlab} through the University of Toronto's Work Study Program. The project I worked on during this time was an online  psychological testing website. I created the website using Node.js, and stored test results using Amazon Web Service's Simple Storage Device. It was challenging to consult with the client about how the testing site should look and run, as there were times when I interpreted the instructions I received differently than how my client intended due to the difference in our backgrounds. Fortunately, these misunderstandings were resolved through further discussion, and through such discussion I was able to broaden my communication abilities. 
% }

% \lettercontent{
%     Another project I have made in Python was a gravity simulator. I used the Pygame module to implement gravity between bodies of various masses. It was interesting to see relationship between the number of bodies and the dynamics of the system. With 2 bodies, the orbits were mainly elliptical, but with 3 bodies and more, the orbits became chaotic. Making the simulator was rewarding as the results matched up with the dynamics I learned in physics class! 
% }
% \lettercontent{
%     I am currently studying Mathematics and Computer Science at the University of Toronto. From the software design courses I have taken, I have gained good object-oriented and test-driven design skills. I have also learned to use Numpy and Matplotlib from machine learning courses. Assignments in several of these courses were written in Python, which coupled with my experience in coding competitions like LeetCode, have greatly heightened my programming and debugging skills, especially in Python. I have also taken a variety of math courses including Real analysis and abstract algebra. My favourite math course is topology! These math courses have vastly increased my problem solving capabilities. 
% }

% Conclude your application letter by thanking the employer for considering you for the position. 
\lettercontent{ 
I am eager to discuss in more detail how my skills align with the requirements of the  \positionnamevalue \hspace{1pt} position. Thank you for considering my application.}
\closing{Sincerely, \\
Leon Yao}
\end{document}  